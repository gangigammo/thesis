近年、TwitterやFacebook等のSNS(ソーシャルネットワークサービス)の普及により、誰もがネットワーク上でのコミュニケーションが可能になったことや、化合物の物性推定の重要性の認識の高まりなどにより、グラフ構造を用いた解析が注目を集めている。グラフとはノードと、二つのノード間を結ぶエッジから構成されている。本研究ではグラフ構造とそれぞれのノードごとに備わっている構造特徴と少数のノードに備わったラベルにより、それ以外のノードのラベルを推定するための手法を提案する。

従来の研究ではグラフ構造において畳み込み演算をするグラフ畳み込みネットワークがラベル推定で高い精度を示している。しかしそのネットワーク1層では各ノードにおいて距離1で隣接したノードからの特徴をとることができず、ラベルの数が少ない場合は精度が低くなってしまう問題がある。そのため、グラフの構造特徴の学習も同時にさせる研究がなされるようになった。構造特徴とはグラフに構造からのみの情報により求められるものであり、主に中心性などが有名である。また、複数の入力のうちどの入力を重要視するかを決定するニューラルネットワークの手法であるattentionメカニズムを用いた手法が、近年グラフニューラルネットワークや画像認識などのさまざまな分野において高い精度を示すことが認識されてきた。

本研究では、グラフ畳み込みネットワークにおける学習に加えて、attentionメカニズムを用いてノードごとの中心性の重要度を測定し、それに比例した構造特徴の学習も両立させるようなモデルを実装した。この手法においては、近隣ノードの学習とグラフ全体の学習を可能にしている。

実験の結果、主にラベルが少ない場合に既存手法を上回る分類精度を得ることが確認できた。
