%\chapter{Introduction}
\chapter{序論}
\section{研究の背景と目的}
近年、TwitterやFacebook等のSNS(ソーシャルネットワークサービス)の普及により、誰もがネットワーク上でのコミュニケーションが可能になったことや、化合物の物性推定の重要性の認識の高まりなどにより、グラフ構造を用いた解析が注目を集めている。グラフとはノードと、二つのノード間を結ぶエッジから構成されている。グラフ構造を用いたデータ解析のタスクには、ノードの分類、ノードのクラスタリング、グラフ分類、リンク予測などがある。

本研究ではグラフ構造とそれぞれのノードごとに備わっている構造特徴と少数のノードに備わったラベルによる学習を用いて、ラベルの与えられていないノードにおけるノードのラベルを予測するようなタスクを行う。このタスクはラベルが分かっているノードとラベルが分かっていないノードのどちらにおいてのノードもグラフ構造と特徴量が分かっているため半教師あり学習という分野の研究である。このタスクはwebページ関係ネットワークのトピックを予測したり、SNS上でのユーザー関心の予測などがある。

このタスクにおいてはグラフベース正則化\cite{LP},\cite{chebnet},\cite{manireg},\cite{semiemb}が古くから研究されている。この手法は、ラベル付きノードに隣接しているノードはそのラベルと等しいラベルを持っているのではないかという推測に基づいた手法であるが、モデルが直感的であり表現力の向上が困難であった。それからグラフの構造をベクトル空間に埋め込むネットワークエンベディング\cite{embedding}の研究が盛んに行われている。この手法を用い、グラフから得たベクトル表現を機械学習モデルで学習させることで、ノード分類やリンク予測において高い精度を得ることが可能になった。この手法の欠点として、タスクによりどのようなベクトル表現を得るべきかを推論することが難しいことが上げられる。

近年、グラフにおいて畳み込み演算を行うグラフ畳み込みネットワークが高い精度を示している。グラフ畳み込み演算にはグラフフーリエ変換を用いる手法と、行列演算を用いて直接的に求める方法の2種類存在する。しかしそのネットワーク1層では各ノードにおいて距離1で隣接したノードからの特徴をとることができず、ラベルの数が少ない場合は精度が低くなってしまう問題がある。そのため、グラフの構造特徴の学習も同時にさせる研究がなされるようになった。構造特徴とはグラフに構造からのみの情報により求められるものであり、主に中心性などが有名である。また、複数の入力のうちどの入力を重要視するかを決定するニューラルネットワークの手法であるattentionメカニズムを用いた手法が、近年グラフニューラルネットワークや画像認識などのさまざまな分野において高い精度を示すことが認識されてきた。

本研究では、グラフ畳み込みネットワークにおける学習に加えて、attentionメカニズムを用いてノードごとの中心性の重要度を測定し、それに比例した構造特徴の学習も両立させるようなモデルを実装した。この手法においては、近隣ノードの学習とグラフ全体の学習を可能にしている。

実験の結果、主にラベルが少ない場合に既存手法を上回る分類精度を得ることが確認できた。

\section{本論文の構成}
本論文は全5章から構成される。第2章では本研究で利用する従来の手法や関連研究について述べる。第3章では本研究における提案手法について具体的に述べる。第4章では論文引用ネットワークを用いた実験を行う。第5章では本論文の結論と今後の課題について述べる。
